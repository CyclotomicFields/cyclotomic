\documentclass{article}
\usepackage[margin=1.2in]{geometry}
\usepackage{parskip}
\usepackage{amsfonts}
\usepackage{amsmath}
\usepackage{amsthm}

\title{Adaptive Algorithms for Arithmetic in Cyclotomic Fields}
\author{Kaashif Hymabaccus, Robert Moore}

\newtheorem{prop}{Proposition}

\begin{document}

\maketitle

Put abstract here.

\section{Introduction}

Currently this is just a place for me to collect some knowledge on
cyclotomic field arithmetic.

\subsection{Picking a basis}

This section is basically taken from the GAP source file {\tt
  cyclotom.c}.

The GAP computer algebra system has an implementation of cyclotomic
field arithmetic. Recall that the field $K_n = \mathbb{Q}(\zeta_n)$
(where $\zeta_n$ is a primitive $n$th root of unity) has dimension
$\varphi(n)$ when seen as a vector space over $\mathbb{Q}$. When
choosing a representation of an element $z \in K_n$, we want this
representation to be unique. This means we have to pick a basis for
$K_n$. The natural one is
$\{ \zeta_n^i : 0 \leq i \leq \varphi(n)-1 \}$. When computing
products and inverses, this would mean we'd essentially be computing
products and inverses in the field $\mathbb{Q}[x]/(\Phi_n(x))$, so
we'd have to compute modulo the polynomial $\Phi_n(x)$.

There is a choice of basis that's easier to work with. Define the
basis $B_n$ to be the set of $\zeta_n^i$ such that:

\begin{itemize}
\item For every odd prime divisor $p$ of $n$,
  $i \notin (n/q)*[-(q/p-1)/2..(q/p-1)/2]$ mod $q$ where $q$ is the
  maximal power of $p$ dividing $n$. For example, if
  $n = 18 = 2 \cdot 3^2$ and $p=3$, then $q=9$.

\item $i \notin (n/q)*[q/2..q-1]$, where $q$ is the maximal power of 2
  in $n$.
\end{itemize}

\begin{prop}
$B_n$ is a basis for $K_n$.
\end{prop}

\begin{proof}
GAP doesn't give a proof, I think the only proof of this is in
German actually. TODO: this probably should be an easy exercise for an
undergraduate i.e. hard for me.

The proof probably has two steps: show that there are $\varphi(n)$
elements, then show that it generates $K_n$ by proving correct the
algorithm to express any element as a linear combination of elements
of $B_n$.
\end{proof}

Give an example for some trivial $n$ and some nontrivial $n$. Prime
and not prime. Squarefree and not squarefree maybe.

Remark: $B_n$ is closed under complex conjugation for odd $n$ but this
isn't possible for even $n$. (TODO: why? maybe a proposition there? or
maybe it's obvious).

\subsection{Rewriting polynomials}

Maybe this should come first. You can rewrite any polynomial in
$\zeta_n^i$ into a linear combination of the $B_n$.

\begin{prop}
  The algorithm for {\tt ConvertToBase} from {\tt cyclotom.c} is
  correct. Here is the algorithm:

  TODO: write out the algorithm in a more functional, easier to read
  way (i.e. not 80s C lmao)
\end{prop}

\begin{proof}
TODO: write a proof of correctness. Or just say ``it's clear''.
\end{proof}

\subsection{Reducing to a minimal subfield}

Suppose we have $z \in K_n$. Then it's possible that $z \in K_m$ where
$m \mid n$, where $m < n$. Reducing the numbers in the representation
is nice.

TODO: should we always go straight to the minimal representation?
Maybe, maybe not.

\subsection{Complexity of arithmetic}

What actually is the complexity of reduction into a minimal cyclotomic
field?

Quote the algorithms from the paper by Johnson, it's a good reference
for ``basic'' polynomial arithmetic.

Addition is probably linear in the number of terms (assuming no
reductions required).

Multiplication is probably linear in the size of each multiplicand. So
quadratic overall probably. But there are usually some reductions
needed.

Equality linear assuming no reductions. But when do you do the
reductions? Is it just before checking equality? When are those
counted?

\subsection{Parallelism}

Is it embarassingly parallel? Improvements over GAP?

\section{Sparse Representations}

\subsection{Vectors}

Put GAP's implementation here. Not the mathy stuff, just talk about
how they use arrays, bags, globals, single-threaded etc.

\subsection{Hash maps}

This is probably my ``naive'' implementation.

\section{Dense Representation}

Probably just the array representations. Maybe some discussion of cool
assembly hacks for multiplying numbers in a single instruction. Or
maybe some cyclic array or list (maybe not linked list for cache
reasons...) thing where certain multiplications can become cheap.

\section{Performance-enhancing heuristics}

This is the real-world, ``street smart not book smart'' section.

\subsection{Chunky and equal-spaced representations}

This is Daniel Roche's thing, we need to implement and cite that.

\subsection{Optimisations for ``simple'' elements}

Work this out as we go. Maybe it will become clear from machine code
analysis. I mean prime order roots, small order roots, maybe orders
with large powers of 2 as factors. Maybe make up some BS measure of
``complexity'' for cyclotomic numbers.

\subsection{Choosing reduction parameters}

When should we reduce the order of an element? By how much? Should we
try to reduce to square-free order fields and then stop? All the way?

Some experiments are very necessary here.

Just do some benchmarks? Some graphs of the speedup? Which algorithms
are most cache-friendly and least susceptible to false sharing?

\section{Benchmarks}

Ours is the best by far! Cherry pick some good benchmarks.

\section{Conclusions and Future Work}

Do it in a GPU lmao. Do machine learning to pick good parameters?
Maybe that's not a buzzword in this case.

Maybe apply it to some research problem and demonstrate that whichever
method turns out to be the best is very good. Blow GAP out of the
water.

Other ideas: apply it to the semidefinite program with symmetries and
some nice representations, groups etc from my Master's thesis so I can
cite it.

\end{document}
