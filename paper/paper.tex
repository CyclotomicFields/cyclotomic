\documentclass{article}
\usepackage[margin=1.2in]{geometry}
\usepackage{parskip}

\title{Adaptive Algorithms for Arithmetic in Cyclotomic Fields}
\author{Kaashif Hymabaccus, Robert Moore}

\begin{document}

\maketitle

\section{Introduction}

\subsection{Background}

\subsection{Overview of existing methods and implementations}

\section{Sparse Representations}

\subsection{Vectors}

Put GAP's implementation here.

\subsection{Hash maps}

This is probably my ``naive'' implementation.

\section{Dense Representation}

Probably just the array representations. Maybe some discussion of cool
assembly hacks for multiplying numbers in a single instruction. Or
maybe some cyclic array or list (maybe not linked list for cache
reasons...) thing where certain multiplications can become cheap.

\section{Performance-enhancing heuristics}

\subsection{Chunky and equal-spaced representations}

This is Daniel Roche's thing, we need to implement and cite that.

\subsection{Optimisations for ``simple'' elements}

Work this out as we go.

\subsection{Choosing reduction parameters}

When should we reduce the order of an element? By how much? Should we
try to reduce to square-free order fields and then stop? All the way?

Some experiments are very necessary here.

\section{Parallelisation}

Just do some benchmarks? Some graphs of the speedup? Which algorithms
are most cache-friendly and least susceptible to false sharing?

\section{Benchmarks}

Ours is the best by far!

\section{Conclusions and Future Work}

Do it in a GPU lmao. Do machine learning to pick good parameters?
Maybe that's not a buzzword in this case.

Maybe apply it to some research problem and demonstrate that whichever
method turns out to be the best is very good. Blow GAP out of the
water.

\end{document}
